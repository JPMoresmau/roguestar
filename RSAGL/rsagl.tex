\documentclass[pdftex]{article}
\usepackage[dvips]{graphicx}
\usepackage[usenames,dvipsnames]{color}
\usepackage[pdftex,colorlinks=true,linkcolor=blue]{hyperref}

\title{ RSAGL Literate Source Document }
\date{ This document was automatically generated from the literate source \today. }

\usepackage{listings}
\lstloadlanguages{Haskell}
\lstnewenvironment{code}
    {\lstset{}%
      \csname lst@SetFirstLabel\endcsname}
    {\csname lst@SaveFirstLabel\endcsname}
    \lstset{
      basicstyle=\small\ttfamily,
      flexiblecolumns=false,
      basewidth={0.5em,0.45em},
      linewidth=6.5in,
      literate={+}{{$+$}}1 {/}{{$/$}}1 {*}{{$*$}}1 {=}{{$=$}}1
               {>}{{$>$}}1 {<}{{$<$}}1 {\\}{{$\lambda$}}1
               {->}{{$\rightarrow$}}2 {>=}{{$\geq$}}2 {<-}{{$\leftarrow$}}2
               {<=}{{$\leq$}}2 {=>}{{$\Rightarrow$}}2 
               {\ .}{{$\circ$}}2 {\ .\ }{{$\circ$}}2
               {>>}{{>>}}2 {>>=}{{>>=}}2
               {>>>}{{>>>}}3 {<<<}{{<<<}}3
               {|}{{$\mid$}}1               
    }

\oddsidemargin 0.0in
\pagestyle{headings}
\hyphenation{Stateful-Arrow State-Arrow Threaded-Arrow Switched-Arrow loop-ed-Dou-bles loop-ed-Con-sec-u-tives}

\begin{document}

\maketitle

\begin{abstract}
The automatically generated documentation to the literate source code of RSAGL, 
the RogueStar Animation and Graphics Library.  This library contains utilities
to generate models and animations targeted at OpenGL using Haskell's HOpenGL
binding.  Roguestar and RSAGL are available at http://roguestar.downstairspeople.org/.
\end{abstract}

\subsection{Contents}

\tableofcontents

\subsection{Audience}

This paper is intended to support people wishing to understand the technical workings of
the RSAGL library.  It is assumed that the reader possesses a fair understanding
of the Haskell programming language, arrows, OpenGL, and linear algebra (as it relates
to affine transformations).

To learn about the Haskell functional programming language, see 
\href{http://www.haskell.org/}{www.haskell.org}.

To learn about arrows, see 
\href{http://www.haskell.org/arrows/}{www.haskell.org/arrows/}.

To learn about OpenGL, see 
\href{http://www.opengl.org/}{www.opengl.org}.  
RSAGL interfaces with OpenGL via the HOpenGL binding.  To learn about HOpenGL, see 
\href{http://www.haskell.org/HOpenGL/}{www.haskell.org/HOpenGL/}.

To learn about linear algebra, a free textbook is available at 
\href{http://joshua.smcvt.edu/linearalgebra/}{joshua.smcvt.edu/linearalgebra/}.

\part{Modeling}

\input{Model.lhs}
\input{Color.lhs}
\input{Material.lhs}
\input{Deformation.lhs}
\input{Extrusion.lhs}
\input{Noise.lhs}
\input{ModelingExtras.lhs}
\input{Optimization.lhs}
\input{Tesselation.lhs}
\input{QualityControl.lhs}
\input{Bottleneck.lhs}

\part{Animation}

The RSAGL FRP arrow transformer is designed to support the functional reactive programming paradigm
specifically with OpenGL.

\input{FRP.lhs}
\input{Edge.lhs}
\input{Scene.lhs}
\input{Animation.lhs}
\input{AnimationExtras.lhs}
\input{RK4.lhs}
\input{Joint.lhs}
\input{KinematicSensors.lhs}
\input{InverseKinematics.lhs}

\part{Arrow Primitives for Functional Reactive Programming}

This section contains arrow transformers used to create the RSAGL FRP arrow.

\input{StatefulArrow.lhs}
\input{SwitchedArrow.lhs}
\input{ThreadedArrow.lhs}
\input{FRPBase.lhs}
\input{ArrowTransformerOptimizer.lhs}

\part{Mathematical Abstractions}

\input{AbstractVector.lhs}
\input{Interpolation.lhs}
\input{Curve.lhs}
\input{CurveExtras.lhs}

\part{Three-Dimensional Mathematics}

\input{Vector.lhs}
\input{Matrix.lhs}
\input{Affine.lhs}
\input{WrappedAffine.lhs}
\input{Homogenous.lhs}
\input{CoordinateSystems.lhs}
\input{Orthagonal.lhs}
\input{BoundingBox.lhs}
\input{RayTrace.lhs}

\part{Specific Numeric Types}

\input{Time.lhs}
\input{Angle.lhs}

\part{License}

\input{../LICENSE}

\end{document}
